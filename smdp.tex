\documentclass[12pt,letterpaper]{article}
%%%%%%%%%%%%%%%%%%%%%%%%%%%%%%
%Force pdflatex processing even with "$ latex" (required by arXiv)
\pdfoutput=1
%%%%%%%%%%%%%%%%%%%%%%%%%%%%%

\usepackage[utf8]{inputenc}
\usepackage{amsmath}
\usepackage{amssymb}
\usepackage{amsthm}
\usepackage{xcolor}
\definecolor{nicered}{rgb}{0.7,0.1,0.1}
\definecolor{nicegreen}{rgb}{0.1,0.5,0.1}
\usepackage[colorlinks=true,citecolor= nicegreen,linkcolor= nicered]{hyperref}
\usepackage{graphicx}
\usepackage{cancel}

\setlength{\textwidth}{180mm}
\setlength{\oddsidemargin}{-1cm}
\setlength{\evensidemargin}{-1cm}
\setlength{\textheight}{22cm} 
\setlength{\topmargin}{-1cm}
\title{Standard Model}

\author{Enrico Nardi}
\begin{document}
\maketitle

\section{Introduction}
%gssi
\section{Unidades naturales}
Usaremos $\hbar=c=1$.  
Nuestro espacio tiempo esta basado en el Grupo $SO(1,3)$, con un factor de conversión entre el factor 1 y e es $c$. 

\section{Repaso del modelo estándar}
Basado en la simetría $SU(3)\times SU(2)\times U(1)$, con 8 gluones, 3 $W$ y bosón gauge abeliano. Cuando la simetría se rompe, tenemos $W_{\mu}; Z_{\mu}$. Como partículas de spin $1/2$, tenemos
\begin{align}
  l=
  \begin{pmatrix}
    \nu_L\\ e_{L}
  \end{pmatrix}\sim& (1,2,-1/2)  & e\sim&(1,1,-1)\\
  Q=
  \begin{pmatrix}
    u_L\\ d_L
  \end{pmatrix}\sim& (3,2,1/6)  & u\sim&(3,1,2/3)\\
&&d\sim&(3,1,-1/3)\\
  H=
  \begin{pmatrix}
    \phi^+\\ \phi^0
  \end{pmatrix}\sim& (1,2,1/2)  & &
\end{align}
\begin{itemize}
\item Los 8 gluones están en la adjunta de $SU(3)$
\item Los 3 $W$ están en la adjunta de $SU(2)$
\item $B_{\mu}$ es singlete de todo.
\end{itemize}
El Lagrangiano es
\begin{align*}
  \mathcal{L}=&-\frac{1}{4}G_{\mu\nu}^{\alpha}-\frac{1}{4}W^{a}_{\mu\nu}W_{a}^{\mu\nu}
-\frac{1}{4}B_{\mu\nu}B^{\mu\nu}\\
&-\frac{g_3^2}{64\pi^2}\theta_3\epsilon_{\mu\nu\rho\sigma}G^{\alpha\mu\nu}G^{\alpha\rho\sigma}\\
&+\text{términos que se pueden eliminar similares al anterior}
\end{align*}
donde
\begin{align*}
  G_{\mu\nu}^{\alpha}=\partial_{\mu}...
\end{align*}
Además
\begin{align*}
  \theta_3=\theta_3+2\pi\,.
\end{align*}

\begin{align*}
  \mathcal{L}_{gf}=&\bar{l}_i \gamma^{\mu}\mathcal{D}_{\mu}l_i
+\bar{e}_i \gamma^{\mu}\mathcal{D}_{\mu}e_i
+\bar{Q}_i \gamma^{\mu}\mathcal{D}_{\mu}Q_i
+\bar{u}_i \gamma^{\mu}\mathcal{D}_{\mu}u_i
+\bar{d}_i \gamma^{\mu}\mathcal{D}_{\mu}d_i
+\bar{N}_i \gamma^{\mu}\partial_{\mu}N_i
\end{align*}
\begin{align*}
  \mathcal{L}_{H}=\left( \mathcal{D}_{\mu}H \right)^{\dagger}
\left( \mathcal{D}^{}{\mu}H \right)-V(H^{\dagger}H)
\end{align*}
\begin{align*}
  \mathcal{L}=-\bar{l}_i y^l_{ij}e_j H
-\bar{Q}_i y^u_{ij}u_j H
-\bar{Q}_i y^d_{ij}d_j \tilde{H}
-\bar{N}_i y^N_{ij}N_j \tilde{H}
\end{align*}
Los parámetros son
\begin{itemize}
\item Gauge sector: $g_1,g_2,g_3,\theta_3$ (4)
\item Yukawa (ME): 3+3+3+3+1=13
\item Higgs: $\mu^2;\lambda$ (2)
\end{itemize}

\section{Problemas del ME}

\begin{align*}
  [\mathcal{L}]=&M^4
\end{align*}
La acción
\begin{align*}
  S=&\int d^4x \mathcal{L} & [S]=&0
\end{align*}
In $D$-dimensions

\begin{align*}
  [\mathcal{L^{(D)}}]=&M^D
\end{align*}
La acción
\begin{align*}
  S=&\int d^Dx \mathcal{L}^{(D)} & [S]=&0
\end{align*}
\begin{align*}
  \left[ \partial_{\mu}\phi \partial^{}{\mu}\phi\right]=&M^D\\
\left[ \bar{\psi} \gamma^{\mu}\psi\right]=& M^D\\
\left[ F^{\mu\nu} F_{\mu\nu}\right]...
\end{align*}
Entonces
\begin{align*}
  lr[\phi^n(\bar{\psi }\psi)^m]=&M^{n+3m}\\
\end{align*}
Otras dimensiones
\begin{itemize}
\item Gauge couplings $[g_j]=0$
\item Yukawa couplings: $[y]=0$
\item Higgs potential parameters: $[\mu]=M^1;\qquad [\lambda]=0$
\end{itemize}
A renormalizable theory
\begin{align*}
  4\ge [\text{dimension of parameters}]\ge 0\,,
\end{align*}
The existence of infinites is required because locality.
\begin{align*}
  \int d^4x\mathcal{L}_{\phi}\sim \int d^4 x\phi^{3}
\end{align*}
Ultaviolet divergences are expected due to locality. In string theory the amplitudes are finite. 

In SM, it is possible to calculate
\begin{align*}
  \mathcal{L}^{(D-5)}_{\mu}\sim&\frac{F_2(0)}{2m_e}\bar{e}\sigma_{\mu\nu}e F^{\mu\nu}\\
&\bar{\mu}(p')\left[ i\frac{\sigma^{\mu\nu}}{2m} \right]\bar{\mu}(p)
\end{align*}


If the fundamental theory is known, the effective Lagrangian terms can be explained in terms of a more fundamental theory. 

But seen to very low energy with only the four light quarks, the Lagrangian does not includes the muon decay. The decay width for $\tau$ decay
\begin{align*}
  \bar{u}_{\nu_{\tau}}(l)\left[ \frac{ig}{\sqrt{2}}\gamma^{\mu}\frac{1}{2}(1-\gamma_5) \right]u_{\tau}(k)\frac{-i}{r^2-M_W^2+i\epsilon}\left[ g_{\mu\nu}+
\frac{(\xi-1)r_{\mu}-r_{\nu}}{r^2-\xi M_W^2} \right]
\bar{u}_j(p)\left[ \frac{ig}{\sqrt{2}}\gamma^{\mu}\frac{1}{2}(1-\gamma_5) \right]v_i(q)
\end{align*}
The effective width is
\begin{align*}
  A_{\tau^{-}\text{decay}}=-i
\frac{g^2}{8M_W^{2}}V_{ij}
\left[\bar{u}_{\nu_{\tau}}\gamma^{\mu}(1-\gamma_5)u_{\tau}  \right]
\left[\bar{u}_j\gamma^{\mu}(1-\gamma_5) v_i  \right]
\end{align*}
So that
\begin{align*}
  \mathcal{L}^{\text{eff}}=-i\frac{G_F}{\sqrt{2}}V_{ij}
\left[\bar{u}_{\nu_{\tau}}\gamma^{\mu}(1-\gamma_5)u_{\tau}  \right]
\left[\bar{u}_j\gamma^{\mu}(1-\gamma_5) v_i  \right]
\end{align*}
is a dimension-six Lagrangian. 

In other situation, where we are now with the SM, were the fundamental
theory is not known. However, we can add to the SM Lagrangian,
effective terms which the same gauge invariance.
\begin{align*}
  \mathcal{L}=\mathcal{L}_{\text{ME}}+\mathcal{L}^{\text{eff}}
\end{align*}
where
\begin{align*}
  \mathcal{L}^{\text{eff}}=\sum_I C_I \theta^{4+n}(\text{light fields})
\end{align*}
where
\begin{align*}
  c_I=\frac{\tilde{c}_I}{\Lambda^n}
\end{align*}
the $\Lambda$-scale is discouraging large.

In Donogue book, a QED calculation is done for $E\ll m_e$ with path integral
\begin{align*}
  Z=\int \left[ D\Phi \right]\left[ D\phi \right]
e^{i\int d^{x}\left[\mathcal{L}(\phi,\phi)+\mathcal{L}(\phi,\Phi)+\mathcal{L}(\Phi,\Phi)  \right]}
\end{align*}
where $\Phi$ are heavy fields, we arrive to
\begin{align*}
  \mathcal{L}^{\text{eff}}=\frac{\alpha^{2}}{90m_e^4}\left[ \left( F_{\mu\nu}F^{\mu\nu} \right)^2
+\frac{7}{16}\left( F_{\mu\nu}\widetilde{F}^{\mu\nu} \right)^2 \right]
\end{align*}
What is the scale of new physics?, without additional parameters the scale of new physics associated with
\begin{align*}
  \frac{1}{\Lambda}(LH)^2\,,
\end{align*}
is $\Lambda\sim 10^{14}$, while the others
\begin{align*}
  \frac{1}{\Lambda^2}\bar{\psi}\psi\bar{\psi}\psi
\end{align*}
the scale is around $100$~TeV

\section{Dimension-0}
Let
\begin{align*}
  \mathcal{L}_0=\frac{1}{2}\partial_{\mu}\varphi\partial^{\mu}\varphi-\frac{1}{2}m^{2}\varphi\varphi-C_{\varphi}
\end{align*}
Therefore
\begin{align*}
  E_0=C_{\varphi}\int_V d^3x+\frac{1}{2}\int
  \frac{d^3\mathbf{p}}{E_{\mathbf{p}}(2\pi)^3}
E_{\mathbf{p}}(2\pi)^3\delta(0)
\end{align*}
Normal ordering:
\begin{align*}
  \left[ a_{\mathbf{p}},a_{\mathbf{p}'}^{*} \right]=2 E_{\mathbf{p}}
(2\pi)^3\delta(\mathbf{p}-\mathbf{p}')
\end{align*}
Divergences appear at $x\to \infty$ and $\mathbf{p}\to\infty$, and
\begin{align*}
  \Omega=\int_V d^3x e^{i(\mathbf{p}=0)}=(2\pi)^3\delta(0)
\end{align*}
with
\begin{align*}
  \rho=E_0/\Omega
\end{align*}
we have
\begin{align*}
  \rho=C_{\varphi}+\frac{1}{16\pi}\left\{
    \underbrace{\Lambda^4+m_{\varphi}\Lambda^2}_{\text{UV}}+
    \underbrace{\frac{1}{4}m^4\log \frac{m^{2}}{\Lambda^2}
+\frac{1}{4}m^4 \left( \frac{1}{2}\log 4 \right)}_{\text{IR
      problem $\propto m_{\varphi}^{2}$}}+m^4\mathcal{O}\left( m^2/\Lambda^2 \right) \right\}
\end{align*}
Including the fermion contribution
\begin{align*}
  E_0^{\psi}=C_{\psi}\int_V d^3x-\sum_{\sigma}\frac{1}{2}\int
  \frac{d^3\mathbf{p}}{E_{\mathbf{p}}(2\pi)^3}
E_{\mathbf{p}}(2\pi)^3\delta(0)
\end{align*}
and
\begin{align*}
  \rho=C_{\psi}-\frac{1}{8\pi}\left\{
    \underbrace{\Lambda^4+m_{\psi}\Lambda^2}_{\text{UV}}+
    \underbrace{\frac{1}{4}m_\psi^4\log \frac{m_\psi^{2}}{\Lambda^2}
+\frac{1}{4}m_\psi^4 \left( \frac{1}{2}\log 4 \right)}_{\text{IR
      problem $\propto m_{\psi}^{2}$}}+m_\psi^4\mathcal{O}\left( m_\psi^2/\Lambda^2 \right) \right\}
\end{align*}

For the full Lagrangian
\begin{align*}
  \rho=C_{\varphi}+C_{\psi}-\frac{1}{16\pi}\left\{
    [m_{\varphi}-m_{\psi}]\Lambda^2+
\cdots  \right\}
\end{align*}
For the top mass (or a SUSY scale around 100~GeV)
\begin{align*}
  \frac{\delta\rho_t}{\rho_{\text{exp}}}\sim 
  \left( \frac{1.7\times 10^{11}\ \text{eV}}{2.3\times 10^{-3}\
      \text{eV}} \right)^4
       \sim 10^{55}
\end{align*}

\section{Dimension-2}

\textbf{Hierarchy Problem}, related with the hierarchy of the Higgs
mass.
\begin{align*}
  V(|\phi|^2)=&\lambda \left( |\phi|^2-v^2 \right)\nonumber\\
            =&\lambda|\phi|^4-\mu^2|\phi|^2
\end{align*}
The contribution from $W,Z,\gamma$ to the Higgs-mass at one-loop is
\begin{align*}
  \frac{1}{16\pi^2}g^2\Lambda^2
\end{align*}
the autointeraction contribution is
\begin{align*}
  \frac{1}{16\pi^2}\lambda^2\Lambda^2
\end{align*}
The top contribution is
\begin{align*}
    -\frac{1}{16\pi^2}\lambda_t^2\Lambda^2
\end{align*}


If $\Lambda_{\text{New Physics}}=10\ $TeV and $m_H=200\ \text{GeV}$
\begin{align}
  \frac{1}{16\pi^2}g^2\Lambda^2\sim &(700\text{GeV})^2\\
  \frac{1}{16\pi^2}\lambda^2\Lambda^2\sim & (500\text{GeV})^2\\
-\frac{1}{16\pi^2}\lambda_t^2\Lambda^2\sim & -(2000\text{GeV})^2\\
\end{align}

and
\begin{align*}
  m_H^2=\left( m_H^0 \right)^2-[-100+10+5]\left( 200\ \text{GeV} \right)^2
\end{align*}

The precise result in terms of the parameter $\mu$ is
\begin{align*}
  \delta \mu^2 =\frac{3}{16\pi^2v^2}
\left[2m_W^2+m_z+m_H-4 m_t^2  \right]\Lambda^2
\end{align*}
the cancellation of the terms inside brackets happen if
\begin{align*}
  m_H^2\sim (307\ \text{GeV})^2
\end{align*}
which is known as the Veltman's condition. 

With RN neutrinos
\begin{align*}
  M_{\nu}=
  \begin{pmatrix}
    0 & \lambda v\\    
    \lambda v &M \\    
  \end{pmatrix}
\end{align*}
with two eigenvalues
\begin{align*}
  m_{\nu}\approx \lambda^2\frac{v^2}{M}\nonumber\\
 M_N\approx M-\lambda^2\frac{v^2}{M}
\end{align*}
for $m\sim 0.1\ $eV, and $\lambda\sim 0.1$
\begin{align*}
  M\sim 10^{12}\ \text{GeV}
\end{align*}
Since RH neutrinos have couplings with the Higgs
\begin{align*}
  -\frac{1}{2}\lambda \overline{N}L\Phi
\end{align*}
The diagram $\Phi$--$\Phi$ at one loop gives the contribution
\begin{align*}
  -i\int
  \frac{d^p}{(2\pi)^4}\frac{i}{\cancel{p}}\frac{i}{\cancel{p}-m}
=&i(i)\int
\frac{\pi^2}{16\pi^4}\frac{1}{2}|p|^2d|p|^2\frac{\cancel{p}_E}{-p_E^2}
\frac{\cancel{p}_E+M}{-p_E^2-M^2}\nonumber\\
=&\frac{-i}{16\pi^2}\int_{M_H}^{\Lambda^{2}}(1-M^2/p^2)dp^{2}\nonumber\\
=&-i \frac{1}{16\pi^2}\left[ \lambda^2-M^2\log \frac{\Lambda^2}{m_H^2} \right]
\end{align*}
As $\Lambda>N$, the Higgs mass acquires a 1-loop contribution of order $M$. 

%%%%%%%%%%%%%%%%%
\bibliographystyle{h-physrev4}%apsrev4-1long
\bibliography{susy}
\end{document}

%%% Local Variables: 
%%% mode: latex
%%% TeX-master: "dmweinberg"
%%% End: 